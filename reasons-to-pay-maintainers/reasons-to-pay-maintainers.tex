\documentclass[12pt,a4paper,notitlepage]{article}
\usepackage{geometry}
\usepackage[yyyymmdd]{datetime}
\usepackage{lmodern}
\usepackage[htt]{hyphenat}
\usepackage{hyperref}
\usepackage{xcolor}
\usepackage[skip=3mm, indent=0mm]{parskip}
\usepackage[sfdefault]{librefranklin}
\usepackage{multicol}
\usepackage{svg}

\geometry{a4paper, left=12mm, top=12mm, bottom=12mm, right=12mm}
\setlength{\columnsep}{12mm}
\renewcommand{\baselinestretch}{1.15}

\definecolor{color-primary}{HTML}{2dde74}
\definecolor{color-secondary}{HTML}{9e86ff}
\definecolor{color-bg}{HTML}{080e17}
\definecolor{color-light-bg}{HTML}{16212d}
\pagecolor{color-bg}
\color{white}
\pagenumbering{gobble}


\begin{document}

\begin{multicols}{2}

\includesvg[width=6cm]{../assets/opensourcepledge-logo-horiz-color@2x.svg}

{\Huge Not Paying Open Source Maintainers Hurts Your Company}

Your company depends on the work of Open Source maintainers. Paying these maintainers helps your company by improving
the stability of your software, increasing innovation, ensuring you comply with new legislation, and attracting
customers, employees and contributors. Not paying maintainers increases your company's exposure to risk. It makes
financial sense for your company to pay the maintainers you rely on, and to join the Open Source Pledge. Here's why.

\textbf{Reason 1:} Paying maintainers allows you to \textbf{prevent stability and security problems} that your company's
software might otherwise suffer from.

If you don't pay for the Open Source software you use, the unpaid maintainers who create it are vulnerable to burnout,
causing your software to become less stable and secure. Exploits such as Log4Shell or the XZ Utils backdoor harm both
the industry generally, and your company specifically. Don't let your business rely on at-risk software — pay the
maintainers who can make your stack sustainable and secure.

\textbf{Reason 2:} Paying maintainers is a way to \textbf{keep the Open Source innovation you rely on from
disappearing}.

Innovative Open Source projects have enabled us to watch YouTube videos, go to space, exchange medical records and keep
in touch with friends and family. But this innovation, which can only happen with an Open Source development model, is
vulnerable if maintainers cannot afford to pay the bills. To continue leveraging innovative Open Source software, pay
the maintainers who do the innovating.

\textbf{Reason 3:} Paying maintainers will help you proactively \textbf{conform to new cybersecurity legislation}.

The EU's Cyber Resilience Act sets out minimum cybersecurity requirements that must be met before software is placed on
the EU market, and you will have to comply with it from December 2027. This requires you to get assurances that the Open
Source software you rely on is secure, and open source software stewards — basically, foundations — are the only ones
that can help. But at least 50\% of foundations say they have insufficient financial support to actually ensure CRA
compliance. Pay the foundations so they can help you comply with the law.

\textbf{Reason 4:} By paying maintainers, you \textbf{show your customers you are a thought leader}.

Being a forward-thinking Open Source pioneer that pays maintainers reflects positively on your company's brand, which
can persuade customers to choose you over a competitor. Paying maintainers tells customers that you are a thought leader
that understands their field deeply and anticipates problems before they occur. To communicate these values to
customers, the Open Source Pledge promotes member companies, be it on the Nasdaq tower in Times Square or in outdoor
advertising campaigns in San Francisco. Join us by paying maintainers.

\textbf{Reason 5}: Paying maintainers allows you to \textbf{build valuable connections with contributors}.

Your company depends on Open Source contributors, whether on those that add features to the Open Source libraries you
depend on, or those who improve Open Source projects your company has published. These contributors allow you to
leverage not only your employees' skills, but also the skills of a global base of specialised developers. Pay them to
make sure they stick around.

\textbf{Reason 6:} Paying maintainers lets you \textbf{make recruiting easier by attracting like-minded employees}.

Developers are aware of the crisis that Open Source sustainability faces. If developers looking for work know that you
pay maintainers because you take seriously the sustainability of the Open Source ecosystem they work within, they'll be
happier working for you than for your competitor. Open Source Pledge members can show off their values by using our
member badges, and showing their job postings on the Open Source Pledge job board.

Convinced? Join the Open Source Pledge.

\end{multicols}

\end{document}
